\section{Motivation}

\section{Challenges Faced}
I quickly found that there were many unanticipated implementation challenges
associated with the the algorithm in the paper. The authors specified it
at a very high level, and although a cursory glance would indicate that it
was fairly straightforward, there were several pieces assumed as given that
needed to be figured out. In this section, I enumerate those problems 
and whether or not I was succesfuly in overcoming them.
\subsection{Convex Programming}


\subsection{Gaussian Width, Diameter of a convex set}
In order to determine the number of iterations $T$ of {\bf Frank-Wolfe},
I needed to compute
\begin{align}
T = \frac{4n \mathrm{diam}(P^{1/2}L)^2}{c(\epsilon, \delta) \ell^\star(L)}
\end{align}
for $L$ an efficient relaxation (a larger convex set that a linear program
can be solved over in polynomial time) of
$\{Au \in \mathbb R^m : u \in B_1 \subset \mathbb R^N\}$ where $A$ was the
query matrix. $4n$ is easy enough as the size of the database, and $c(\epsilon,
\delta)$ comes from the needed width of the Gaussian mechanism:
\[ c(\epsilon, \delta) = \frac{1 + \sqrt{2 \ln(1/\delta)}}\epsilon \]
For the diameter $\mathrm{diam}(P^{1/2}L)$, by the author's {\bf Lemma 4.4}
we see that $\mathrm{diam}(P^{1/2}L) \leq 1$. 
Note that $l^\star$ is the Gaussian width, described as
\[ \ell^\star(X) = \mathbb E_{g \sim \mathcal N(0, 1)^m} \; \max_{v \in X} \;
        \langle g, v \rangle \]
For the particular $L$ described on the bottom of page 11 (and $L_0$) described
above, the author's claim $\ell^\star(P^{1/2}L_0) \leq
    d^{\left \lceil k/2 \right \rceil/2}$ and that there is a constant
$C$ such that \[\ell^\star(QL) \leq \ell^\star(QL_0)\] for all
$Q \in \mathbb R^{d^k \times d^k}$, in particular $P^{1/2}$ and the identity.
Making a variation on their computation of $\ell^\star(P^{1/2}L)$ in 
{\bf Lemma 4.1}, we see that for for $w, z \in \{\pm 1\}^{d^{k/2}}$, we have

$||w \otimes z||^2_2 = \sum 1 = d^k$, so that by the stability of Gaussians
if $g \sim \mathcal N(0, 1)$ then $\langle g, w \otimes z \rangle \sim \mathcal
N(0, d^{k/2})$
Taking the maximum of $n$ normal random variables with variance $\sigma^2$
is $O(\sigma \sqrt{\log(n)})$, and as $\sigma = d^{k/2}, n = 2^{2d^{k/2}}$ we
have that it is $O(d^{k/2})$, so that $\ell^\star(L) \approx d^{3k/4}$.
So then I computed $T$ as 
\[ T = \frac{4n}{c(\epsilon, \delta) d^{3k/4}} \]
\section{Architecture and Design}

\section{Results and Tests}

\section{Suggestions for Future Work}


